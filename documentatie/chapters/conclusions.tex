\chapter*{Concluzii finale} 
\addcontentsline{toc}{chapter}{Concluzii}

Lucrarea "Home Smartify - soluție pentru controlul dispozitivelor și securitatea casei" încearcă să ușureze modul de viață al utilizatorului, reușind să minimizeze legătura dintre acesta și dispozitivele pe care le are în casă. Aceasta este posibilă datorită aplicației de mobil pe care o poate folosi să își controleze senzorii din orice parte a lumii. Peste 6 miliarde de persoane folosesc un telefon mobil\footnote{\url{https://www.oberlo.com/statistics/how-many-people-have-smartphones}}, făcându-l un prim candidat când vine vorba despre modul în care utilizatorul să își controleze casa.

Cererea de soluții de tipul Smart Home este în continuă creștere. În anul 2022 numărul de case smart a crescut cu 14\%\footnote{\url{https://todayshomeowner.com/smart-home/guides/smart-home-facts-and-statistics/}} datorită 

\section{Direcții pe viitor}

Consider că proiectul poate fi îmbunătățit în mai multe arii. O primă remediere ar fi înlocuirea plăcii Raspberry Pi Zero W cu placa Raspberry Pi 4 deoarece procesorul cu un singur nucleu al primei plăci nu poate face față unui val de cereri API. Nucleele adiționare vor crește viteza de procesare a datelor care sunt citite de la senzori și vor scădea timpul de așteptare între cererile făcute de către telefonul utilizatorului.

Un alt punct de îmbunătățire ar fi modulul de transmitere al datelor senzorilor statici.

Un alt punct ce ar putea fi corectat ar fi afișarea celor mai recente informații din baza de date fără ca utilizatorul să fie nevoit să apese pe butonul de reîmprospătare. Acest lucru este posibil ca la fiecare citire recent introdusă în baza de date, serverul să notifice fiecare telefon de acest lucru și să trimită informațiile actualizate.

Nu în ultimul rând, aș hosta serverul în cloud pentru ca alți utilizatori să poată folosi această tehnologie smart.