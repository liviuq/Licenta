\chapter*{Introducere} 
\addcontentsline{toc}{chapter}{Introducere}

Lucrarea "\textbf{Home Smartify - soluție pentru controlul dispozitivelor și securitatea casei}" aduce în prim plan o abordare inovatoare pentru a facilita și îmbunătăți modul de viață al utilizatorului în cadrul casei sale. În era tehnologiei inteligente, unde conectivitatea și controlul la distanță devin tot mai importante, această soluție oferă consumatorului posibilitatea de a avea controlul complet asupra dispozitivelor și senzorilor din propria casă, indiferent de locația în care se află. Securitatea datelor utilizatorului este cea mai mare prioritate în dezvoltarea acestei aplicații, fiind stocate în totalitate în mediul local. O abordare inovatoare și ușor de utilizat pentru tehnologia Smart Home este propusă în această lucrare, care contribuie semnificativ la îmbunătățirea atât a confortului, cât și a securității locuinței.

Structura proiectului este afișată în următorul tabel ce încearcă să ajute la înțelegerea în detaliu a aplicației prin descrierea sumară a fiecărui capitol.

\begin{itemize}
	\item Capitolul I. Prezentarea unor aplicații similare utilizate în domeniul tehnologiei Smart Home, cum ar fi Home Assistant, Apple Home și IFTTT.
	
	\item Capitolul II. Descrierea tehnologiilor utilizate în implementarea soluției, cum ar fi Flutter SDK, Flask, Raspberry Pi Zero W și microcontrolerele Arduino Nano și ESP8266.
	
	\item Capitolul III. Prezentarea implementării soluției Home Smartify ce conține aplicația de mobil cu paginile principale, modul debug, precum și elemente legate de Application Programming Interface și
	echipamentul senzorilor.
	
	\item Capitolul IV. Descrierea scenariilor de utilizare ale aplicației, cum ar fi colectarea de date, comutarea luminilor din holul intrării, generarea raportului și detectarea problemelor în comunicare.
	
	\item Concluzii. Imbunatatiri pe viitor
\end{itemize}