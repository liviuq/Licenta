\chapter*{Motivație} 
\addcontentsline{toc}{chapter}{Motivație}

Lucrarea de licență intitulată "\textbf{Home Smartify - soluție pentru controlul dispozitivelor și securitatea casei}" prezintă cum prin integrarea dispozitivelor \textbf{"Internet of things"}\footnote{\url{https://www.techtarget.com/iotagenda/definition/Internet-of-Things-IoT}.}, se poate ajunge la un nivel ridicat de control asupra locuinței personale: monitorizarea mișcărilor dintr-o casă, conectarea/deconectarea unui consumator de la alimentare, toate fiind ușor de executat cu ajutorul aplicației mobile.

Home Smartify a fost creată datorită dorinței personale de a-mi monitoriza si controla apartamentul într-un mod ușor, intuitiv și privat. Jucându-mă cu plăcile \textbf{Raspberry Pi}\footnote{\url{https://www.raspberrypi.org/}.} împreună cu diferiți senzori, am reușit să creez un sistem ce securiza ușa camerei mele cu ajutorul unui cod numeric, roboți ce urmează un traseu predefinit și prize inteligente. Problema pe care am întâmpinat-o a fost că nu exista o entitate ce să comande fiecare modul în parte, eu fiind responsabil de a trimite informațiile necesare către modulul respectiv. Observând utilitatea acestei funcționalități și faptul că mă distrez în timp ce programez și conectez componentele între ele prin fire, am decis să realizez un sistem ce poate \emph{"vorbi"} cu aceste module, putând fi comandate din aplicația de telefon.

Inițial, ideea a fost concepută doar cu latura de monitorizare, folosind un singur tip de senzori. Oferit pe post de feedback de către domnul profesor coordonator Cristian Vidrașcu, am realizat că implementarea unei modalități de control a acestor dispozitive adaugă o nouă arie de funcționalitate și creativitate în gestionarea sarcinilor din casă.
