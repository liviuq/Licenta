\chapter*{Contribuții}
Soluțiile de a transforma o casă într-un domeniu \emph{inteligent} cresc anual în număr, astfel am optat pentru a crea una originală, de la aplicația de mobil, până la modulele hardware pe care le-am folosit.

Aplicația de mobil "Home Smartify" a luat naștere ca răspuns al întrebării \emph{"Ce pot face pentru a-mi automatiza casa?"}. Având la îndemână diferite componente și senzori, am decis să creez o aplicație de telefon care să controleze aceste dispozitive de la distanță, oferind astfel utilizatorului posibilitatea de a-și automatiza și personaliza experiența în propria casă. Prin integrarea dispozitivelor și senzorilor, soluția propusă facilitează \textbf{controlul asupra locuinței}, permițând \textbf{monitorizarea mișcărilor}, \textbf{conectarea/ deconectarea consumatorilor} și \textbf{ajustarea parametrilor de mediu}, cum ar fi iluminatul, într-un mod \textbf{ușor și intuitiv}. În plus, este pus un accent pe \textbf{securitatea datelor utilizatorului}, asigurând \textbf{stocarea lor în totalitate în mediul local}, fapt ce contribuie la protejarea informațiilor personale și confidențialitatea utilizatorului.

Aș dori să mulțumesc domnului profesor coordonator Cristian Vidrașcu pentru libertatea alegerii temei licenței și profesionalismul de care a dat dovadă.
